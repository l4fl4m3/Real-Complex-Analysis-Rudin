\documentclass{article}
\usepackage{amsfonts} 
\usepackage{amsmath}
\usepackage{amssymb}
\usepackage{dcolumn}
\usepackage{tikz-cd}

\newcolumntype{2}{D{.}{}{4.0}}
\title{Solutions: Real and Complex Analysis by Walter Rudin}
\author{Hassaan Naeem}
\date{\today}
\begin{document}
\maketitle


\section*{Chapter 1. Abstract Integration}
\subsection*{Exercise 1}
Does there exist an infinite $\sigma$-algebra which has only countably many members?
\\\\
\textbf{Solution:}
No. Impossible.

\subsection*{Exercise 2}
Prove an analog of Theorem 1.8 for $n$ functions.
\\\\
\textbf{Solution:}
We have that $u_1, u_2, ..., u_n$ are real measurable functions on a measurable space.
\\\\We let $f(x) = (u_1(x), u_2(x), ..., u_n(x))$.
Since $h = \Phi \circ f$, Theorem 1.7 shows that it is enough to prove measurability of $f$.
\\\\We let $B = I_1 \times I_2 \times ... \times I_n$. We then have $f(B) = (u_1(I_1), u_2(I_2)), ... , u_n(I_n)$.
We then have that $f^{-1}(B) = u_1^{-1}(I_1) \cap u_2^{-1}(I_2) \cap ... \cap u_n^{-1}(I_n)$,
which is measurable by our measurability assumption on $u_1, u_2, ..., u_n$.
\\\\Every open set $V$ in $I_1 \times I_2 \times ... \times I_n$ is a countable union of such $B$ which we call $B_i$.
Hence we have that $f^{-1}(V)=f^{-1}(\bigcup\limits_{i=1}^{\infty} B_i) = \bigcup\limits_{i=1}^{\infty} f^{-1}(B_i)$.
Hence $f^{-1}(V)$ is measurable. $\quad \square$

\subsection*{Exercise 3}
Prove that if $f$ is a real function on a measurable space $X$ such that $\{ x:f(x) \ge r\}$ is measurable
for every rational $r$, then $f$ is measurable.
\\\\
\textbf{Solution:}
We know that $f$ is measurable if for every open set $V$ in $\mathcal{O}_{std}$, $f^{-1}(V)$ is measurable set.
Here $\mathcal{O}_{std}:\{ (a,b): a < x < b \ \forall x \in \mathbb{R} \}$ is the standard topology on $\mathbb{R}$ and is just the collection of all open intervals $(a,b)$.
We know that $\{x \in X: f(x) \ge q\}$ is a measurable set $ \forall q \in \mathbb{Q}$. Since we know that $\mathbb{Q}$ is a dense subset of $\mathbb{R}$, we can always get arbitrarily close to any $r \in \mathbb{R}$.
We let $\forall r \in \mathbb{R}, \ (q_n)_{n\in\mathbb{N}}$ be a decreasing sequence in $\mathbb{Q}$ such that $\lim_{n\to\infty} q_n = r$.
We then have that $\{x \in X: f(x) > r\} = \bigcup_{n=1}^{\infty} \{ x \in X: f(x) > q_n\}$.
By definition, the right hand side is measurable, hence every $r$ is measurable.
Hence, for every open interval in $ I \in \mathcal{O}_{std}$, $f^{-1}(I)$ is a measurable set, hence $f$ is measurable.


\subsection*{Exercise 4}
Let $\{a_n\}$ and $\{b_n\}$ be sequences in $[-\infty, \infty]$, and prove the following assertions:

\begin{enumerate}
    \item[(a)]  \[\limsup_{n\to\infty} (-a_n) = - \liminf_{n\to\infty} a_n \]
    \item[(b)]  \[\limsup_{n\to\infty} (a_n + b_n) \le \limsup_{n\to\infty} a_n + \limsup_{n\to\infty} b_n \]\\
    provided none of the sums is of the form $\infty - \infty$.
    \item[(c)] If $a_n \le b_n$ for all $n$, then 
        \[ \liminf_{n\to\infty} a_n \le \liminf_{n\to\infty} b_n\]
\end{enumerate}
Show by an example that strict inequality can hold for (b).
\\\\
\textbf{Solution:}

\subsection*{Exercise 5}
\\\\
\textbf{Solution:}

\subsection*{Exercise 6}
\\\\
\textbf{Solution:}

\subsection*{Exercise 7}
\\\\
\textbf{Solution:}

\subsection*{Exercise 8}
\\\\
\textbf{Solution:}

\subsection*{Exercise 9}
\\\\
\textbf{Solution:}

\subsection*{Exercise 10}
\\\\
\textbf{Solution:}

\subsection*{Exercise 11}
\\\\
\textbf{Solution:}

\subsection*{Exercise 12}
\\\\
\textbf{Solution:}

\subsection*{Exercise 13}
\\\\
\textbf{Solution:}

\end{document}
