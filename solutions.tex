\documentclass{article}
\usepackage{amsfonts} 
\usepackage{amsmath}
\usepackage{amssymb}
\usepackage{dcolumn}
\usepackage{tikz-cd}

\newcolumntype{2}{D{.}{}{4.0}}
\title{Solutions: Real and Complex Analysis by Walter Rudin}
\author{Hassaan Naeem}
\date{\today}
\begin{document}
\maketitle


\section*{Chapter 1. Abstract Integration}
\subsection*{Exercise 1}
Does there exist an infinite $\sigma$-algebra which has only countably many members?
\\\\
\textbf{Solution:}
No. Impossible.

\subsection*{Exercise 2}
Prove an analog of Theorem 1.8 for $n$ functions.
\\\\
\textbf{Solution:}
We have that $u_1, u_2, ..., u_n$ are real measurable functions on a measurable space.
\\\\We let $f(x) = (u_1(x), u_2(x), ..., u_n(x))$.
Since $h = \Phi \circ f$, Theorem 1.7 shows that it is enough to prove measurability of $f$.
\\\\We let $B = I_1 \times I_2 \times ... \times I_n$. We then have $f(B) = (u_1(I_1), u_2(I_2)), ... , u_n(I_n)$.
We then have that $f^{-1}(B) = u_1^{-1}(I_1) \cap u_2^{-1}(I_2) \cap ... \cap u_n^{-1}(I_n)$,
which is measurable by our measurability assumption on $u_1, u_2, ..., u_n$.
\\\\Every open set $V$ in $I_1 \times I_2 \times ... \times I_n$ is a countable union of such $B$ which we call $B_i$.
Hence we have that $f^{-1}(V)=f^{-1}(\bigcup\limits_{i=1}^{\infty} B_i) = \bigcup\limits_{i=1}^{\infty} f^{-1}(B_i)$.
Hence $f^{-1}(V)$ is measurable. $\quad \square$

\subsection*{Exercise 3}
Prove that if $f$ is a real function on a measurable space $X$ such that $\{ x:f(x) \ge r\}$ is measurable
for every rational $r$, then $f$ is measurable.
\\\\
\textbf{Solution:}
We know that $f$ is measurable if for every open set $V$ in $\mathcal{O}_{std}$, $f^{-1}(V)$ is measurable set.
Here $\mathcal{O}_{std}:\{ (a,b): a < x < b \ \forall x \in \mathbb{R} \}$ is the standard topology on $\mathbb{R}$ and is just the collection of all open intervals $(a,b)$.
We know that $\{x \in X: f(x) \ge q\}$ is a measurable set $ \forall q \in \mathbb{Q}$. Since we know that $\mathbb{Q}$ is a dense subset of $\mathbb{R}$, we can always get arbitrarily close to any $r \in \mathbb{R}$.
We let $\forall r \in \mathbb{R}, \ (q_n)_{n\in\mathbb{N}}$ be a decreasing sequence in $\mathbb{Q}$ such that $\lim_{n\to\infty} q_n = r$.
We then have that $\{x \in X: f(x) > r\} = \bigcup_{n=1}^{\infty} \{ x \in X: f(x) > q_n\}$.
By definition, the right hand side is measurable, hence every $r$ is measurable.
Hence, for every open interval in $ I \in \mathcal{O}_{std}$, $f^{-1}(I)$ is a measurable set, hence $f$ is measurable.


\subsection*{Exercise 4}
Let $\{a_n\}$ and $\{b_n\}$ be sequences in $[-\infty, \infty]$, and prove the following assertions:

\begin{enumerate}
    \item[(a)]  \[\limsup_{n\to\infty} (-a_n) = - \liminf_{n\to\infty} a_n \]
    \item[(b)]  \[\limsup_{n\to\infty} (a_n + b_n) \le \limsup_{n\to\infty} a_n + \limsup_{n\to\infty} b_n \]\\
    provided none of the sums is of the form $\infty - \infty$.
    \item[(c)] If $a_n \le b_n$ for all $n$, then 
        \[ \liminf_{n\to\infty} a_n \le \liminf_{n\to\infty} b_n\]
\end{enumerate}
Show by an example that strict inequality can hold for (b).
\\\\
\textbf{Solution:}
\\\\
a) 
\begin{equation*}
    \begin{aligned}
        \limsup_{n\to\infty} (-a_n) &= \inf_{n\ge0} \sup_{m\ge n} (-a_m)\\
        &= \inf_{n\ge0} (- \inf_{m\ge n} a_m)\\
        &= - \sup_{n\ge0} \inf_{m\ge n} a_m\\
        & = - \liminf_{n\to\infty} (a_n)\\
    \end{aligned}
\end{equation*}
\\
b)
We let $a_n = \sup_{m \ge n} a_m$ and $ b_n = \sup_{m \ge n} b_m$.
It is trivial that $A \subseteq B \implies \sup A \le \sup B$.
We can then observe that $\forall m \ge n,\ a_n \ge a_m$ and $b_n \ge b_m$. Hence we have that: 
\begin{equation*}
    \begin{aligned}
        a_n + b_n &\ge a_m + b_m \\
        \sup_{m \ge n} (a_n + b_n) = a_n + b_n &\ge \sup_{m \ge n} (a_m + b_m)\\
        \lim_{n\to\infty} (a_n + b_n) &\ge \lim_{n\to\infty}\sup_{m \ge n} (a_m + b_m)\\
        \lim_{n\to\infty} (\sup_{m \ge n} a_m + \sup_{m \ge n} b_m) &\ge \lim_{n\to\infty}\sup_{m \ge n} (a_m + b_m)\\
        \lim_{n\to\infty}\sup_{m \ge n} a_m + \lim_{n\to\infty}\sup_{m \ge n} b_m &\ge \lim_{n\to\infty}\sup_{m \ge n} (a_m + b_m)\\
        \limsup_{n\to\infty} a_n + \limsup_{n\to\infty} b_n &\ge \limsup_{n\to\infty}(a_n +b_n) \quad \square   
    \end{aligned}
\end{equation*}
\\
c)
We have that $\forall n$:
\begin{equation*}
    \begin{aligned}
        \inf_{m \ge n} a_m &\le \inf_{m \ge n} b_m\\
    \end{aligned}
\end{equation*}
Where for the sequences $(\inf_{m \ge n} a_m)_{n\in\mathbb{N}}$ and $(\inf_{m \ge n} b_m)_{n\in\mathbb{N}}$ we have:
\begin{equation*}
    \begin{aligned}
        \lim_{n\to\infty}\inf_{m \ge n} a_m &\le \lim_{n\to\infty}\inf_{m \ge n} b_m\\
        \liminf_{n\to\infty} a_n &\le \liminf_{n\to\infty} b_n \quad \square
    \end{aligned}
\end{equation*}
\subsection*{Exercise 5}
(a) Suppose $f: X \rightarrow [-\infty, \infty]$ and $g:X \rightarrow [-\infty, \infty]$ are measurable. Prove that the sets $ \{ x: f(x) < g(x) \}, \{ x: f(x) = g(x)\}$ are measurable.
\\\\
(b) Prove that the set of points at which a sequence of measurable real-valued functions converges
(to a finite limit) is measurable. 
\\\\
\textbf{Solution:}
\\\\
a) Since $f$ and $g$ are measurable, from Excercise 3, we can deduce that $\forall q \in \mathbb{Q}, \ \{x:f(x) \ge q\}$ and $\{x:g(x) \ge q\}$ are measurable sets. Hence their complements and strict inequality conditioned sets
$\{x:f(x) < q\}, \{x:g(x) < q\}, \{x:f(x) > q\}, \{x:g(x) > q\}, \{x:g(x) \le q\}, \{x:g(x) \le q\}$ are also measurable. We then have that $\{x:f(x) < g(x)\}, \{x: f(x) = g(x)\} & \iff \{x:f(x) \le g(x)\} = X$. Then we have:
\begin{equation*}
    \begin{aligned}
        X^C = \{x:f(x) > g(x)\} &= \bigcup\limits_{q\in\mathbb{Q}}\left\{\{x:f(x) > q\} \cap \{x:g(x) < q\} \right\}\\
            &= \bigcup\limits_{q\in\mathbb{Q}}\left\{\{x:f(x) \le q\} \cup \{x:g(x) \ge q\} \right\}
    \end{aligned}
\end{equation*}
which measurable since it is the union of countably many measurable sets. 
Hence we have that $(X^C)^C = X$ is measurable.
\\\\
b) If we have a sequence of real-valued measurable functions $(f_n)_{n\in\mathbb{N}}$ which converge to a finite limit say $a$,
we know that:
\[ \lim_{n\to\infty} f_n = a \iff \liminf_{n\to\infty} f_n = \limsup_{n\to\infty} f_n = a \]
By \textbf{Theorem 1.14} we know that $h = \limsup_{n\to\infty}f_n$ and $g = \sup_{n\ge 1} f_n$ are measurable.
From these it follows that $\inf_{n\ge1}f_n$ and $\liminf_{n\to\infty}f_n$ are measurable.
Then we have that $\{ x: \liminf_{n\to\infty} f_n = \limsup_{n\to\infty} \} = X$ and:
\begin{equation*}
    \begin{aligned}
        X^C = \{ x: \liminf_{n\to\infty} f_n > \limsup_{n\to\infty} \} \cup \{ x: \liminf_{n\to\infty} f_n < \limsup_{n\to\infty} \}\\
    \end{aligned}
\end{equation*}
which we know by (a) to be measurable. Hence $X$ is measurable.

\subsection*{Exercise 6}
Let $X$ be an uncountable set, let $\mathfrak{M}$ be the collection of all sets $E \subset X$ such that
such that either $E$ or $E^C$ is at most countable, and define $\mu(E) = 0$ in the first case, $\mu(E)=1$ in the
second. Prove that $\mathfrak{M}$ is a $\sigma$-algebra in $X$ and that $\mu$ is a measure on $\mathfrak{M}$. Describe
the corresponding measurable functions and their integrals.
\\\\
\textbf{Solution:}
By defintion $X,  \emptyset \in \mathfrak{M}$. Additionally if $E \in \mathfrak{M}$, then $E^C \in \mathfrak{M}$.
We then let $E = \cup_{i=1}^{\infty} E_i$ where $E_i \in \mathfrak{M}$. If all $E_i$ are countable then we have that $E$ is
also countable, since a countable union of countable sets is countable, hence $E \in \mathfrak{M}$. If $\exists E_{i_u}$ such that $E_{i_u}^C$ is countable,
then $E^C = \cap_{i=1}^{\infty}E_i^C \subseteq E_{i_u}^C$, and hence $E^C$ is countable, hence $E \in \mathfrak{M}$. Hence $\mathfrak{M}$ is a $\sigma$-algebra in $X$.
\\\\
We then look at $\mu$. We let $E = \cup_{i=1}^{\infty}E_i$ be a countable collection of pairwise disjoint sets $E_i \in \mathfrak{M}$. Similar to before
if all $E_{i}$ are countable then $E$ is countable and hence $\Sigma_{i=1}^{\infty}\mu(E_i) = \Sigma_{i=1}^{\infty} 0 = 0= \mu(E)$.
If $\exists E_{i_{k}}$ such that $E_{i_{k}}^C$ is countable, then $E^C$ is countable. Hence $\Sigma_{i=1}^{\infty}\mu(E_i) = \Sigma_{i=1, i \neq i_k}^{\infty} \mu(E_i) + \mu(E_{i_k}) = \Sigma_{i=1}^{\infty}0 + 1 = 1 = \mu(E)$.
Hence we have that $\mu(\cup_{i=1}^{\infty}E_i) = \Sigma_{i=1}^{\infty}\mu(E_i)$. Hence $\mu$ is a measure on $\mathfrak{M}$.
\\\\
Let $f: X \rightarrow R$, where $R$ is an arbitrary range. We know that if $f$ is measurable then either $f^{-1}(x)$ or $(f^{-1}(x))^C$ is measurable and hence countable.
If we have that $(f^{-1}(x))^C$ is countable then by the measure $\mu$ that we have $f(x)$ is almost constant.

\subsection*{Exercise 7}
Suppose $f_n:X \to [0, \infty]$ is measurable for $n=1,2,3,..., f_1 \ge f_2 \ge f_3 \ge ... \ge 0, \ f_n(x) \to f(x)$ as $n \to \infty$,
for every $x \in X$, and $f_1 \in L^1(\mu)$. Prove that then
\[ \lim_{n\to\infty} \int_X f_n \,d\mu = \int_X f \,d\mu \]
and show that this conclusion does \textit{not} follow if the condition ``$ f_1 \in L^1(\mu)$'' is omitted.
\\\\
\textbf{Solution:}
If $\forall x \in X, \ f_1(x) < \infty $.
We have that $\lim_{n\to\infty}f_n(x) = f(x)$ where each $f_n$ is measurable on $X$. We also have that $f_1(x) \ge f_n(x) \ge 0$ hence $|f_n(x)| \le f_1(x) \in L^1(\mu)$.
By Lebesgue's Dominated Convergence Theorem (LDCT), the conclusion then follows.
\\\\ 
Else we can let $E = \{ x \in X : f_1(x) = \infty\}$. If we assume that $\mu(E) > 0$, then we also have that $\int_X|f_1|d\mu = \infty$, but since $f_1 \in L^1(\mu)$, this is
a contradiction, hence $\mu(E) = 0$. Therefore we have that $\int_X|f_1|d\mu = \int_{X\setminus E}|f_1|d\mu + \int_E|f_1|d\mu = \int_{X\setminus E}|f_1|d\mu < \infty$.
Hence the conclusion again follows by above.
\\\\
As seen above if ``$f_1\in L^1(\mu)$'' is omitted, the conclusion does not follow.  

\subsection*{Exercise 8}
Put $f_n = \chi_E$ if $n$ is odd, $f_n=1-\chi_E$ if $n$ is even. What is the relavance of this example to Fatou's Lemma ?
\\\\
\textbf{Solution:}

\subsection*{Exercise 9}
Suppose $\mu$ is a positive measure on $X, f: X \to [0,\infty]$ is measurable, $\int_X f d\mu = c$, where $0<c<\inf$,
and $\alpha$ is a constant. Prove that
\[ \lim_{n\to\infty} \int_X n \log [1+(f/n)^\alpha]\ d\mu = 
    \begin{cases} 
        \infty & \text{if} \ 0<\alpha<1\\
        c & \text{if} \ \alpha = 1\\
        0 & \text{if} \ 1<\alpha<\infty
    \end{cases}
\]
\\\\
\textbf{Solution:}

\subsection*{Exercise 10}
Suppose $\mu(X)<\infty$, $\{f_n\}$ is a sequence of bounded complex measurable functions on $X$,
and $f_n \to f$ uniformly on $X$. Prove that
\[ \lim_{n\to\infty} \int_X f_n \,d\mu  = \int_X f_n \,d\mu\]
and show that the hypothesis ``$\mu(X) < \infty$'' cannot be omitted.
\\\\
\textbf{Solution:}
Since we have a sequence of bounded complex measurable functions, we have that $|f_n(x)| \le g(x) $, where $g(x)$ is some bound.
Since $\mu(X) < \infty$, then $\int_X |g(x)| < \infty$, hence $g(x) \in L^1(\mu)$.
By uniform convergence, we also have that $f(x) = \lim_{n\to\infty} f_n(x)$. The conclusion then follows by LDCT.
\\\\
If ``$\mu(X) < \infty$'' is omitted, then we can have $\int_X |g(x)| = \infty$, and hence $g(x) \notin L^1(\mu)$ and hence LDCT does not apply,
and the conclusion does not follow.
\subsection*{Exercise 11}
Show that
\[ A = \bigcap_{n=1}^{\infty}\bigcup_{k=n}^{\infty} E_k\]
in Theorem 1.41, and hence prove the theorem without any reference to integration.
\\\\
\textbf{Solution:}
If $x \in \bigcup_{k \in K} E_k$ then $x$ is in at least one of the $E_k$. If $x \in \bigcap_{k \in K} E_k$ then $x$ is in all of the $E_k$.
Hence we can conclude that $\bigcap_{n=1}^{\infty}\bigcup_{k=n}^{\infty} E_k$ are all the elements in either $E_n, E_{n+1}, E_{n+2}, ...$, without regard for the
the size of $n$. Hence this must be the same as being part of infinitely many of the $E_k$.
\\\\
More formally, if $x$ is in infinitely many $E_k$, then for all $n=1,2,3,..., \infty \  \exists \, k> n: x\in E_k$.
Hence $x \in \bigcup_{k=n}^{\infty} E_k$, however it is so for all $n$, hence it is in the interesection of all such unions,
hence $x \in \bigcap_{n=1}^{\infty}\bigcup_{k=n}^{\infty} E_k$. For the other direction, it is enough to show that if $x$ is not in
infinitely many $E_k$ then $A \neq \bigcap_{n=1}^{\infty}\bigcup_{k=n}^{\infty} E_k$. Hence, if $x$ is not in infinitely many $E_k$, 
then there is some largest $k$ and hence $E_k$ in which $x$ lies.
Therefore, $\exists \, n>k: x\notin \bigcup_{k=n}^{\infty} E_k$. Hence $x \notin \bigcap_{n=1}^{\infty}\bigcup_{k=n}^{\infty} E_k \quad \square$
\\\\
We let $F_n = \cup_{k=n}^{\infty} E_k$. From this it is obvious that $F_{n+1} \subset F_n$. 
We then have that 
\begin{equation*}
    \begin{aligned}
        \mu(A) &= \mu(\bigcap_{n=1}^{\infty}F_n)\\
        &= \lim_{n\to\infty} \mu(F_n) \quad (\text{Theorem 1.19 (e)})\\
        &= \lim_{n\to\infty} \mu(\bigcup_{k=n}^{\infty}E_k)\\
        &\le \lim_{n\to\infty} \sum_{k=n}^{\infty} \mu(E_k) = 0\\
    \end{aligned}
\end{equation*}
since we know that $\sum_{k=1}^{\infty}\mu(E_k) < \infty$, as well as the fact that the tail must go to 0 (Theorem 3.23, Principles of Mathematical Analysis; Rudin).

\subsection*{Exercise 12}
Suppose $f \in L^1(\mu)$. Prove that to each $\epsilon > 0$ there exists a $\delta>0$ such that $\int_E |f| \,d\mu < \epsilon$
whenever $\mu(E) < \delta$.
\\\\
\textbf{Solution:}
Since $f\in L^1(\mu)$ we know that $\int_E |f| \, d\mu$ is finite.
By definition $\int_E |f| \, d\mu = \sup \sum_{i=1}^{\infty} \alpha_i \mu(A_i \cap E)$.
In addition we know that $\mu(E) < \delta$. Therefore for each $\epsilon$ we can choose $\delta = \frac{\epsilon}{\sum_{i=1}^{n} \alpha_i}$.
hence we have:
\begin{equation*}
    \begin{aligned}
        \int_E |f| \, d\mu &= \sup \sum_{i=1}^{n} \alpha_i \mu(A_i \cap E)\\
        &\le \sum_{i=1}^{n} \alpha_i \mu(A_i \cap E)\\
        &\le \sum_{i=1}^{n} \alpha_i \mu(E)  = \mu(E) \sum_{i=1}^{n} \alpha_i = \delta \sum_{i=1}^{n} \alpha_i = \epsilon \\
    \end{aligned}
\end{equation*}
\subsection*{Exercise 13}
Show that proposition 1.24(c) is also true when $c = \infty$.
\\\\
\textbf{Solution:}
If $f = 0$, then we have that $\int_E cf \,d\mu = \int_E \infty \cdot 0 \,d\mu = \int_E 0 \,d\mu = 0 = \infty \cdot 0 = \infty \cdot \int_E 0 \,d\mu =  c \int_E f \,d\mu$.
\\\\
If $f > 0$, we know that $\infty \cdot f = \infty \ \forall f>1$.
Then we have that if $\mu(E) = 0$ then $\int_E cf \,d\mu = 0 =  c \int_E f \,d\mu$.
If $\mu(E) \neq 0$, then $\int_E cf \,d\mu = \int_E \infty \cdot f \,d\mu = \int_E \infty \,d\mu = \infty = \infty \cdot \int_E f \,d\mu = c\int_E f \,d\mu$.
test


\section*{Chapter 2. Positive Borel Measures}
\subsection*{Exercise 1}
Let $\{f_n\}$ be a sequence of real nonnegative functions on $\mathbb{R}^1$, and consider the following four statements:
\begin{enumerate}
    \item[(a)]  If $f_1$ and $f_2$ are upper semicontinuous, then $f_1 + f_2$ is upper semicontinuous.
    \item[(b)]  If $f_1$ and $f_2$ are lower semicontinuous, then $f_1 + f_2$ is lower semicontinuous.
    \item[(c)]  If each $f_n$ is upper semicontinuous, then $ \sum_1^{\infty} f_n$ is upper semicontinuous.
    \item[(d)]  If each $f_n$ is lower semicontinuous, then $ \sum_1^{\infty} f_n$ is lower semicontinuous.
\end{enumerate}
Show that three of these are true and that one is false. What happens when the word ``nonnegative'' is omitted? Is the truth of the
statements affected? Is the truth of the statements affected if $\mathbb{R}^1$ is replaced by a general topological space?
\\\\
\textbf{Solution:}
If $f_1(x)$ and $f_2(x)$ are upper semicontinuous, we work with the following set $\{ x: f_1(x) + f_2(x) < \alpha\}$, then:
\begin{equation*}
    \begin{aligned}
        \{ x: f_1(x) + f_2(x) < \alpha\} &= \{ x: f_1(x) < \alpha - f_2(x) \}\\
        &= \bigcup_{q \in \mathbb{Q}} (\{x:f_1(x) < q < \alpha - f_2(x)\})\\
        &= \bigcup_{q \in \mathbb{Q}} (\{x:f_1(x) < q\} \cap \{x:f_2(x) < \alpha - q\})
    \end{aligned}
\end{equation*}
In the last equality the two sets on the right are open, given their upper continuity. Then, the intersection of open sets are open, as well as their union.
Hence we that $\{ x: f_1(x) + f_2(x) < \alpha\}$ is open, hence $f_1(x) + f_2(x)$ is upper semicontinuous.
\\\\
By similar argument, if $f_1(x)$ and $f_2(x)$ are lower semicontinuous, then:
\begin{equation*}
    \begin{aligned}
        \{ x: f_1(x) + f_2(x) > \alpha\} &= \{ x: f_1(x) > \alpha - f_2(x) \}\\
        &= \bigcup_{q \in \mathbb{Q}} (\{x:f_1(x) > q > \alpha - f_2(x)\})\\
        &= \bigcup_{q \in \mathbb{Q}} (\{x:f_1(x) > q\} \cap \{x:f_2(x) > \alpha - q\})
    \end{aligned}
\end{equation*}
is open, and hence $f_1(x)$ + $f_2(x)$ is lower semicontinuous.
\\\\
If each $f_n(x)$ is lower semicontinuous, then:
\begin{equation*}
    \begin{aligned}
        \{ x: \sum_1^\infty f_n(x) > \alpha\} &= \{ x: \lim_{N\to\infty} \sum_{n=1}^N f_n(x) > \alpha \}\\
        &= \{ x: \sup_{N \ge 1} \sum_{n=1}^N f_n(x) > \alpha \} \\
    \end{aligned}
\end{equation*}
Previously, we have shown that a finite sum of lower semicontinuous functions is lower semicontinuous.
By \textbf{Definition 2.8 (c)}, we have that supremum of any collection of lower semicontinuous functions is lower semicontinuous. Hence the we get that the above is lower semicontinuous.
\\\\
If each $f_n(x)$ is upper semicontinuous, then firstly let:
\[ f_n = 
    \begin{cases} 
        1 & \text{if} \ x \ge \frac{1}{n}\\
        0 & \text{if} \ x < \frac{1}{n}\\
    \end{cases}
\]
Then we have that each $f_n$ is upper semicontinuous. However we have:
\[ \sup_{n} f_n = 
    \begin{cases} 
        1 & \text{if} \ x > 0\\
        0 & \text{if} \ x \le 0\\
    \end{cases}
\]
which is not upper semicontinuous.
Then by a similar argument as before we have:
\begin{equation*}
    \begin{aligned}
        \{ x: \sum_1^\infty f_n(x) < \alpha\} &= \{ x: \lim_{N\to\infty} \sum_{n=1}^N f_n(x) < \alpha \}\\
        &= \{ x: \sup_{N \ge 1} \sum_{n=1}^N f_n(x) < \alpha \} \\
    \end{aligned}
\end{equation*}
Again we know that a finite sum of upper semicontinuous functions is upper semicontinuous.
However we have just shown that for our $f_n$ the supremum is not necessarily upper semicontinuous.
Hence the the implication does not follow.
\subsection*{Exercise 2}
Let $f$ be an arbitrary complex function on $\mathbb{R}^1$, and define
\begin{equation*}
    \begin{aligned}
        \phi(x, \delta) &= \sup \{ |f(s)-f(t)|:s,t \in (x-\delta, x+\delta) \}, \\
        \phi(x) &= \inf \{ \phi(x, \delta): \delta > 0 \} \\
    \end{aligned}
\end{equation*}
Prove that $\phi$ is upper semicontinuous, that $f$ is continuous at a point $x$ if and only if
$\phi(x)=0$, and hence that the set of points of continuity of an arbitrary complex function
is a $G_\delta$. Formulate and prove an analogous statement for general topological spaces
in place of $\mathbb{R}^1$.
\\\\
\textbf{Solution:}
We choose arbitrary $x_a \in \mathbb{R}$. By the second equation we then let $y > \phi(x_a)$.
Hence there must be a $\delta_a >0: \phi(x_a, \delta_a)<y$.
We then let $x \in (x_a-\frac{\delta_a}{2}, x_a+\frac{\delta_a}{2})$ and $s,t \in (x-\frac{\delta_a}{2}, x+ \frac{\delta_a}{2})$.
Hence we must have that $(x-\frac{\delta_a}{2}, x+\frac{\delta_a}{2}) \subseteq (x_a-\delta_a,x_a+\delta_a)$, since
$x_a - \delta_a < x-\frac{\delta_a}{2} < x+\frac{\delta_a}{2} < x_a+\delta_a$.
Hence $\phi(x, \frac{\delta_a}{2}) \le \phi(x_a, \delta_a) < y$ (due to supremums).
Hence for any $\delta$, we have that $\inf\{\phi(x,\delta):\delta>0\} \le \phi(x_a, \delta_a) < y$.
Hence we have that $\forall x \in (x_a-\frac{\delta_a}{2}, x_a+\frac{\delta_a}{2}), \ \phi(x) < y$.
Notice that $x_a$ and $y$ were arbitrary.
Hence $\phi$ is upper semicontinuous.
\\\\
\subsection*{Exercise 3}
Let $X$ be a metric space, with metric $\rho$.
For any nonempty $E\subset X$, define
\[
    \rho_E(x) = inf\{ \rho(x,y): y\in E \}
\]
Show that $\rho_E$ is a uniformly continuous function on $X$. If $A$ and $B$ are disjoint nonempty closed
subsets of $X$, examine the relevance of the function
\[
    f(x) = \frac{\rho_A(x)}{\rho_A(x)+\rho_B(x)}
\]
to Urysohn's lemma.
\\\\
\textbf{Solution:}
We know from the definiton that $\forall y\in E, x,z \in X$
\[ \rho_E(x) \le \rho(x,y) \le \rho(x,z) + \rho(z,y) \]
\[ \rho_E(z) \le \rho(z,y) \le \rho(z,x) + \rho(x,y) \]
Hence we get that $\rho_E(x) \le \rho(x,z) + \rho_E(z)$ and $\rho_E(z) \le \rho(z,x) + \rho_E(y)$.
Hence $\rho_E(x) - \rho_E(z) \le \rho(x,z)$ and $\rho_E(z) - \rho_E(x) \le \rho(x,z)$.
\\\\
Hence $\forall x,z \in X \ |\rho_E(z) - \rho_E(x)| \le \rho(x,z)$, and we have uniform continuity.
\\\\
Since the $A$ and $B$ are closed and disjoint, the function shown is a mapping to the closed interval.
We have that $f: X \rightarrow [0,1]$. Since $f(x) = 1 \ \forall x \in B$ and $f(x) \neq 0 \ \forall x \in A^C$, meaning its support lies in $A^C$, by Urysohn's Lemma, we have that if $B$ is compact
then $B \prec f \prec A^C$.

\subsection*{Exercise 4}
Examine the proof of the Reisz representation theorem and prove the following two statements:

\begin{enumerate}
    \item[(a)]  If $E_1 \subset V_1$ and $E_2 \subset V_2$, where $V_1$ and $V_2$ are disjoint open sets,
                then $\mu(E_1 \cup E_2) = \mu(E_1) + \mu(E_2)$, even if $E_1$ and $E_2$ are not in $\mathfrak{M}$.
    \item[(b)]  If $E \in \mathfrak{M}_F$, then $E = N \cup K_1 \cup K_2 \cup ...$, where $\{K_i\}$ is a disjoint countable
                collection of compact sets and $\mu(N) = 0$.
\end{enumerate}
\\\\
\textbf{Solution:}
\\\\
(a) Let $U$ be an open cover of $V_1$ and $V_2$.
Hence we have, $\mu(U) \ge \mu(U \cap (V_1 \cup V_2)) \ge \mu(U \cap V_1) + \mu(U \cap V_2) \ge \mu(E_1) +  \mu(E_2)$.
\\\\
Then we take the infimum, yielding
\begin{equation*}
    \begin{aligned}
        \inf_{E_1\cup E_2 \subset U, U open} \mu(U) & \ge  \inf_{E_1\cup E_2 \subset U, U open} (\mu(E_1) +  \mu(E_2))\\
        \inf_{E_1\cup E_2 \subset U, U open} \mu(U) & \ge \mu(E_1) +  \mu(E_2)\\
        \mu(E_1 \cup E_2) & \ge \mu(E_1) +  \mu(E_2)
    \end{aligned}
\end{equation*}
Also we have by subadditivity that $\mu(E_1 \cup E_2) \le \mu(E_1) + \mu(E_2)$.
Hence $\mu(E_1 \cup E_2) = \mu(E_1) + \mu(E_2)$.
\\\\
(b) We let $E_0 = E$. From the proof, we have that $K_1 \subset E_0 \subset V_1$, where $K_1$ is compact
and $V_1$ is open, and $\mu(V_1 - K_1)<\epsilon$. Hence we have that $K_1 \in \mathfrak{M}_F$. Again by the proof, we have that $E_1 - K_1 \in \mathfrak{M}_F$.
Then, let $E_1 = E_0-K_1$, and similarly we have $K_2 \subset E_1 \subset V_2$... Hence we can always find
$K_n \subset E_{n-1}\subset V_n$, where $K_n$ is compact and $V_n$ is open, and $\mu(V_n - K_n)<\epsilon$. Also, $K_n, (E_{n-1}-K_{n}) \in \mathfrak{M}_F$.
We let $\epsilon = \frac{1}{n}$ and $E_n = E_{n-1} - K_n$. We also let $N = E - \cup_{n=1}^{\infty}K_n$, where $N\in \mathfrak{M}_F$.
We have that $N \subseteq E_n$, hence $\mu(N) \le \mu(E_n) = \mu(E_{n-1} - K_n) \le \mu(V_n - K_n) < \frac{1}{n}$.
Taking the limit, we have $\mu(N) = 0$. And from before we have, $E = N \cup (\cup_{n=1}^{\infty}K_n)$.

\subsection*{Exercise 5}
Let $E$ be Cantor's familiar ``middle thirds'' set.
Show that $m(E)=0$, even though $E$ and $R^1$ have the same cardinality.
\\\\
\textbf{Solution:}

\end{document}
