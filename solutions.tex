\documentclass{article}
\usepackage{amsfonts} 
\usepackage{amsmath}
\usepackage{amssymb}
\usepackage{dcolumn}
\usepackage{tikz-cd}

\newcolumntype{2}{D{.}{}{4.0}}
\title{Solutions: Real and Complex Analysis by Walter Rudin}
\author{Hassaan Naeem}
\date{\today}
\begin{document}
\maketitle


\section*{Chapter 1. Abstract Integration}
\subsection*{Exercise 1}
Does there exist an infinite $\sigma$-algebra which has only countably many members?
\\\\
\textbf{Solution:}
No. Impossible.

\subsection*{Exercise 2}
Prove an analog of Theorem 1.8 for $n$ functions.
\\\\
\textbf{Solution:}
We have that $u_1, u_2, ..., u_n$ are real measurable functions on a measurable space.
\\\\We let $f(x) = (u_1(x), u_2(x), ..., u_n(x))$.
Since $h = \Phi \circ f$, Theorem 1.7 shows that it is enough to prove measurability of $f$.
\\\\We let $B = I_1 \times I_2 \times ... \times I_n$. We then have $f(B) = (u_1(I_1), u_2(I_2)), ... , u_n(I_n)$.
We then have that $f^{-1}(B) = u_1^{-1}(I_1) \cap u_2^{-1}(I_2) \cap ... \cap u_n^{-1}(I_n)$,
which is measurable by our measurability assumption on $u_1, u_2, ..., u_n$.
\\\\Every open set $V$ in $I_1 \times I_2 \times ... \times I_n$ is a countable union of such $B$ which we call $B_i$.
Hence we have that $f^{-1}(V)=f^{-1}(\bigcup\limits_{i=1}^{\infty} B_i) = \bigcup\limits_{i=1}^{\infty} f^{-1}(B_i)$.
Hence $f^{-1}(V)$ is measurable. $\quad \square$

\subsection*{Exercise 3}
Prove that if $f$ is a real function on a measurable space $X$ such that $\{ x:f(x) \ge r\}$ is measurable
for every rational $r$, then $f$ is measurable.
\\\\
\textbf{Solution:}
We know that $f$ is measurable if for every open set $V$ in $\mathcal{O}_{std}$, $f^{-1}(V)$ is measurable set.
Here $\mathcal{O}_{std}:\{ (a,b): a < x < b \ \forall x \in \mathbb{R} \}$ is the standard topology on $\mathbb{R}$ and is just the collection of all open intervals $(a,b)$.
We know that $\{x \in X: f(x) \ge q\}$ is a measurable set $ \forall q \in \mathbb{Q}$. Since we know that $\mathbb{Q}$ is a dense subset of $\mathbb{R}$, we can always get arbitrarily close to any $r \in \mathbb{R}$.
We let $\forall r \in \mathbb{R}, \ (q_n)_{n\in\mathbb{N}}$ be a decreasing sequence in $\mathbb{Q}$ such that $\lim_{n\to\infty} q_n = r$.
We then have that $\{x \in X: f(x) > r\} = \bigcup_{n=1}^{\infty} \{ x \in X: f(x) > q_n\}$.
By definition, the right hand side is measurable, hence every $r$ is measurable.
Hence, for every open interval in $ I \in \mathcal{O}_{std}$, $f^{-1}(I)$ is a measurable set, hence $f$ is measurable.


\subsection*{Exercise 4}
Let $\{a_n\}$ and $\{b_n\}$ be sequences in $[-\infty, \infty]$, and prove the following assertions:

\begin{enumerate}
    \item[(a)]  \[\limsup_{n\to\infty} (-a_n) = - \liminf_{n\to\infty} a_n \]
    \item[(b)]  \[\limsup_{n\to\infty} (a_n + b_n) \le \limsup_{n\to\infty} a_n + \limsup_{n\to\infty} b_n \]\\
    provided none of the sums is of the form $\infty - \infty$.
    \item[(c)] If $a_n \le b_n$ for all $n$, then 
        \[ \liminf_{n\to\infty} a_n \le \liminf_{n\to\infty} b_n\]
\end{enumerate}
Show by an example that strict inequality can hold for (b).
\\\\
\textbf{Solution:}
\\\\
a) 
\begin{equation*}
    \begin{aligned}
        \limsup_{n\to\infty} (-a_n) &= \inf_{n\ge0} \sup_{m\ge n} (-a_m)\\
        &= \inf_{n\ge0} (- \inf_{m\ge n} a_m)\\
        &= - \sup_{n\ge0} \inf_{m\ge n} a_m\\
        & = - \liminf_{n\to\infty} (a_n)\\
    \end{aligned}
\end{equation*}
\\
b)
We let $a_n = \sup_{m \ge n} a_m$ and $ b_n = \sup_{m \ge n} b_m$.
It is trivial that $A \subseteq B \implies \sup A \le \sup B$.
We can then observe that $\forall m \ge n,\ a_n \ge a_m$ and $b_n \ge b_m$. Hence we have that: 
\begin{equation*}
    \begin{aligned}
        a_n + b_n &\ge a_m + b_m \\
        \sup_{m \ge n} (a_n + b_n) = a_n + b_n &\ge \sup_{m \ge n} (a_m + b_m)\\
        \lim_{n\to\infty} (a_n + b_n) &\ge \lim_{n\to\infty}\sup_{m \ge n} (a_m + b_m)\\
        \lim_{n\to\infty} (\sup_{m \ge n} a_m + \sup_{m \ge n} b_m) &\ge \lim_{n\to\infty}\sup_{m \ge n} (a_m + b_m)\\
        \lim_{n\to\infty}\sup_{m \ge n} a_m + \lim_{n\to\infty}\sup_{m \ge n} b_m &\ge \lim_{n\to\infty}\sup_{m \ge n} (a_m + b_m)\\
        \limsup_{n\to\infty} a_n + \limsup_{n\to\infty} b_n &\ge \limsup_{n\to\infty}(a_n +b_n) \quad \square   
    \end{aligned}
\end{equation*}
\\
c)
We have that $\forall n$:
\begin{equation*}
    \begin{aligned}
        \inf_{m \ge n} a_m &\le \inf_{m \ge n} b_m\\
    \end{aligned}
\end{equation*}
Where for the sequences $(\inf_{m \ge n} a_m)_{n\in\mathbb{N}}$ and $(\inf_{m \ge n} b_m)_{n\in\mathbb{N}}$ we have:
\begin{equation*}
    \begin{aligned}
        \lim_{n\to\infty}\inf_{m \ge n} a_m &\le \lim_{n\to\infty}\inf_{m \ge n} b_m\\
        \liminf_{n\to\infty} a_n &\le \liminf_{n\to\infty} b_n \quad \square
    \end{aligned}
\end{equation*}
\subsection*{Exercise 5}
(a) Suppose $f: X \rightarrow [-\infty, \infty]$ and $g:X \rightarrow [-\infty, \infty]$ are measurable. Prove that the sets $ \{ x: f(x) < g(x) \}, \{ x: f(x) = g(x)\}$ are measurable.
\\\\
(b) Prove that the set of points at which a sequence of measurable real-valued functions converges
(to a finite limit) is measurable. 
\\\\
\textbf{Solution:}
\\\\
a) Since $f$ and $g$ are measurable, from Excercise 3, we can deduce that $\forall q \in \mathbb{Q}, \ \{x:f(x) \ge q\}$ and $\{x:g(x) \ge q\}$ are measurable sets. Hence their complements and strict inequality conditioned sets
$\{x:f(x) < q\}, \{x:g(x) < q\}, \{x:f(x) > q\}, \{x:g(x) > q\}, \{x:g(x) \le q\}, \{x:g(x) \le q\}$ are also measurable. We then have that $\{x:f(x) < g(x)\}, \{x: f(x) = g(x)\} & \iff \{x:f(x) \le g(x)\} = X$. Then we have:
\begin{equation*}
    \begin{aligned}
        X^C = \{x:f(x) > g(x)\} &= \bigcup\limits_{q\in\mathbb{Q}}\left\{\{x:f(x) > q\} \cap \{x:g(x) < q\} \right\}\\
            &= \bigcup\limits_{q\in\mathbb{Q}}\left\{\{x:f(x) \le q\} \cup \{x:g(x) \ge q\} \right\}
    \end{aligned}
\end{equation*}
which measurable since it is the union of countably many measurable sets. 
Hence we have that $(X^C)^C = X$ is measurable.
\\\\
b) If we have a sequence of real-valued measurable functions $(f_n)_{n\in\mathbb{N}}$ which converge to a finite limit say $a$,
we know that:
\[ \lim_{n\to\infty} f_n = a \iff \liminf_{n\to\infty} f_n = \limsup_{n\to\infty} f_n = a \]
By \textbf{Theorem 1.14} we know that $h = \limsup_{n\to\infty}f_n$ and $g = \sup_{n\ge 1} f_n$ are measurable.
From these it follows that $\inf_{n\ge1}f_n$ and $\liminf_{n\to\infty}f_n$ are measurable.
Then we have that $\{ x: \liminf_{n\to\infty} f_n = \limsup_{n\to\infty} \} = X$ and:
\begin{equation*}
    \begin{aligned}
        X^C = \{ x: \liminf_{n\to\infty} f_n > \limsup_{n\to\infty} \} \cup \{ x: \liminf_{n\to\infty} f_n < \limsup_{n\to\infty} \}\\
    \end{aligned}
\end{equation*}
which we know by (a) to be measurable. Hence $X$ is measurable.

\subsection*{Exercise 6}
Let $X$ be an uncountable set, let $\mathfrak{M}$ be the collection of all sets $E \subset X$ such that
such that either $E$ or $E^C$ is at most countable, and define $\mu(E) = 0$ in the first case, $\mu(E)=1$ in the
second. Prove that $\mathfrak{M}$ is a $\sigma$-algebra in $X$ and that $\mu$ is a measure on $\mathfrak{M}$. Describe
the corresponding measurable functions and their integrals.
\\\\
\textbf{Solution:}
By defintion $X,  \emptyset \in \mathfrak{M}$. Additionally if $E \in \mathfrak{M}$, then $E^C \in \mathfrak{M}$.
We then let $E = \cup_{i=1}^{\infty} E_i$ where $E_i \in \mathfrak{M}$. If all $E_i$ are countable then we have that $E$ is
also countable, since a countable union of countable sets is countable, hence $E \in \mathfrak{M}$. If $\exists E_{i_u}$ such that $E_{i_u}^C$ is countable,
then $E^C = \cap_{i=1}^{\infty}E_i^C \subseteq E_{i_u}^C$, and hence $E^C$ is countable, hence $E \in \mathfrak{M}$. Hence $\mathfrak{M}$ is a $\sigma$-algebra in $X$.
\\\\
We then look at $\mu$. We let $E = \cup_{i=1}^{\infty}E_i$ be a countable collection of pairwise disjoint sets $E_i \in \mathfrak{M}$. Similar to before
if all $E_{i}$ are countable then $E$ is countable and hence $\Sigma_{i=1}^{\infty}\mu(E_i) = \Sigma_{i=1}^{\infty} 0 = 0= \mu(E)$.
If $\exists E_{i_{k}}$ such that $E_{i_{k}}^C$ is countable, then $E^C$ is countable. Hence $\Sigma_{i=1}^{\infty}\mu(E_i) = \Sigma_{i=1, i \neq i_k}^{\infty} \mu(E_i) + \mu(E_{i_k}) = \Sigma_{i=1}^{\infty}0 + 1 = 1 = \mu(E)$.
Hence we have that $\mu(\cup_{i=1}^{\infty}E_i) = \Sigma_{i=1}^{\infty}\mu(E_i)$. Hence $\mu$ is a measure on $\mathfrak{M}$.
\subsection*{Exercise 7}
\\\\
\textbf{Solution:}

\subsection*{Exercise 8}
\\\\
\textbf{Solution:}

\subsection*{Exercise 9}
\\\\
\textbf{Solution:}

\subsection*{Exercise 10}
\\\\
\textbf{Solution:}

\subsection*{Exercise 11}
\\\\
\textbf{Solution:}

\subsection*{Exercise 12}
\\\\
\textbf{Solution:}

\subsection*{Exercise 13}
\\\\
\textbf{Solution:}

\end{document}
